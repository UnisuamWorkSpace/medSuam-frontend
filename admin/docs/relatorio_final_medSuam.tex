\documentclass[
    12pt,
    openright,
    oneside,
    a4paper,
    chapter=TITLE,
    brazil
]{abntex2}

% =============================
% Pacotes básicos
% =============================
\usepackage{lmodern}
\usepackage[T1]{fontenc}
\usepackage[utf8]{inputenc}
\usepackage{microtype}
\usepackage{graphicx}
\usepackage{color}
\usepackage{float}
\usepackage{indentfirst}
\usepackage{titlesec}
\usepackage{hyperref}
\usepackage{amsmath, amsfonts, amssymb}

% Configuração de hyperlinks
\hypersetup{
    colorlinks=true,
    linkcolor=blue,
    urlcolor=blue,
    citecolor=black,
    pdftitle={Relatório Final do Projeto MedSuam},
    pdfauthor={Gabriel Ferreira Dantas; Igor da Silva Sant'anna; Ronivaldo Domingues de Andrade}
}

% =============================
% Informações do relatório
% =============================
\titulo{Relatório Final do Projeto -- MedSuam}
\autor{
Gabriel Ferreira Dantas \\
Igor da Silva Sant\textquoteright anna \\
Ronivaldo Domingues de Andrade
}
\local{Rio de Janeiro -- RJ}
\data{2025}
\instituicao{
Centro Universitário Augusto Motta -- UNISUAM \\
Curso Superior de Análise e Desenvolvimento de Sistemas \\
Projeto de Desenvolvimento em Back-End
}
\tipotrabalho{Relatório Técnico}

\hypersetup{hidelinks}

\begin{document}

\imprimircapa
\imprimirfolhaderosto*

\tableofcontents*

\pagestyle{abntheadings}

% =====================================================================
\chapter{Introdução}
%\addcontentsline{toc}{chapter}{Introdução}

O presente relatório tem como finalidade documentar, de forma clara e estruturada, o desenvolvimento das quatro sprints que compõem o projeto realizado na disciplina de Projeto de Back-end II. Cada sprint teve como objetivo a construção progressiva de funcionalidades essenciais para um sistema seguro, escalável e alinhado às boas práticas de desenvolvimento.

Na \textbf{Sprint 1}, foi estabelecida a base do sistema, incluindo a modelagem do banco de dados, o cadastro de usuários, o login e as primeiras validações de segurança. Essa etapa preparou a infraestrutura fundamental sobre a qual as demais funcionalidades foram construídas.

A \textbf{Sprint 2} concentrou-se no desenvolvimento de um módulo administrativo completo. Nessa fase, foram implementados níveis hierárquicos de permissão (Admin e Super Admin), além de um CRUD robusto para gerenciamento de usuários, garantindo controle e segurança sobre operações administrativas.

A \textbf{Sprint 3} ampliou significativamente a segurança do sistema, introduzindo a autenticação em dois fatores (2FA), desenvolvida por meio da integração com a API SendGrid. Essa funcionalidade adicionou uma camada verificadora adicional ao processo de login, reforçando a confiabilidade e proteção dos acessos.

Por fim, a \textbf{Sprint 4} abordou o fluxo completo de redefinição de senha, envolvendo validações front-end, envio de códigos de verificação, tratamento seguro de sessões e atualização criptografada de credenciais.

Assim, este relatório apresenta detalhadamente os objetivos, as atividades realizadas, as funcionalidades implementadas e os resultados alcançados em cada sprint, seguido de uma conclusão que sintetiza os aprendizados e a evolução técnica adquirida durante o desenvolvimento do projeto.


% =====================================================================
\chapter{Sprint 1 – Estrutura Inicial do Sistema de Autenticação}

\section{Objetivos}
A Sprint 1 teve como objetivo construir a base do sistema de autenticação, incluindo o cadastro de usuários (pacientes e médicos), login funcional e preparação da estrutura necessária para recuperação de senha.

\section{Atividades Executadas}
Durante esta etapa foram utilizados PHP, MySQL, HTML, CSS, JavaScript e o banco \texttt{bd\_medsuam.sql}. Seguindo orientações da disciplina, foram aplicadas boas práticas como sanitização de dados, verificação de e-mails duplicados e o uso da função \texttt{password\_hash()} para proteção das senhas.

A primeira tarefa consistiu na criação e organização do banco de dados. O script inicial criou tabelas como: \texttt{paciente}, \texttt{medico}, \texttt{telefone}, \texttt{endereco}, \texttt{rg}, \texttt{especialidade}, entre outras que se fizeram necessárias. A conexão com o banco foi centralizada no arquivo \texttt{dbMedsuam.php}.

\subsection{Cadastro de Pacientes}
O cadastro de pacientes sanitiza os dados com \texttt{mysqli\_real\_escape\_string()}, valida campos obrigatórios e bloqueia o registro caso o e-mail já exista nas tabelas de pacientes ou médicos. Após validação, o sistema insere os dados do paciente e registra endereço, telefone e RG vinculados ao seu ID.

\subsection{Cadastro de Médicos}
O cadastro de médicos segue lógica semelhante, com criptografia de senha e registro de telefone e especialidade após validações.

\subsection{Login}
O login foi implementado no arquivo \texttt{userpage.php}. As credenciais são verificadas no banco, e uma sessão segura é iniciada em caso de sucesso, permitindo acesso a áreas internas restritas.

\section{Resultados Alcançados}
A Sprint 1 permitiu criar uma base sólida para o restante do sistema. O grupo adquiriu experiência prática com criação de CRUD básico, organização de dados no banco, integração entre front-end e back-end, sanitização de dados, proteção de senhas e controle de sessões.

% =====================================================================
\chapter{Sprint 2 – Desenvolvimento do CRUD Administrativo}

\section{Objetivos}
A Sprint 2 teve como objetivo desenvolver um módulo administrativo completo, permitindo que usuários com privilégios especiais gerenciassem os demais usuários do sistema.

\section{Atividades Executadas}
\begin{itemize}
	\item Criação do diretório \texttt{admin/} na raiz do projeto para conter os códigos do Admin, fazendo com que o acesso ao painel admin seja pela rota \textit{localhost/medsuam/admin} em caso de uso local.
    \item Implementação do CRUD completo (Criar, Ler, Atualizar e Deletar).
    \item Criação dos níveis de permissão: \textit{Admin} e \textit{Super Admin}.
    \item Desenvolvimento dos controles de acesso para cada tipo de administrador.
    \item Implementação da lógica de bloqueio para impedir que administradores comuns gerenciem outros admins.
\end{itemize}

\section{Funcionalidades Implementadas}
\begin{itemize}
    \item Cadastro, listagem, edição e exclusão de usuários.
    \item Sistema de hierarquia:
    \begin{itemize}
        \item \textbf{Super Admin}: acesso total.
        \item \textbf{Admin}: acesso total, exceto sobre perfis administrativos.
    \end{itemize}
    \item[\textbf{OBS.:}] Não foi implementado um script específico para inserção do primeiro usuário Super Admin, nesse caso foi criado o banco de dados já com ele inserido ou inserido manualmente no PHPmyAdmin quando se necessitava de testes, a criação desse script foi deixada para uma futura implementação real do sistema.
\end{itemize}

\section{Resultados Alcançados}
O módulo administrativo foi concluído com sucesso, oferecendo controle interno robusto, seguro e totalmente funcional.

% =====================================================================
\chapter{Sprint 3 – Autenticação de Dois Fatores (2FA)}

\section{Objetivos}
Implementar um sistema de autenticação em dois fatores para aumentar a segurança no processo de login.

\section{Atividades Executadas}
\begin{itemize}
    \item Integração da API SendGrid para envio de códigos de verificação. A escolha do SendGrid se dá pela facilidade de implementação.
    \item Criação da página de verificação pós-login.
    \item Implementação do fluxo completo de geração, envio e validação do código.
\end{itemize}

\section{Funcionalidades Implementadas}
\textbf{OBS.:} Nessa parte do projeto, foi usada duas bibliotecas prontas, que foram instaladas via \texttt{composer}, são elas: a \texttt{Dotenv\textbackslash Dotenv} e \texttt{SendGrid\textbackslash Mail\textbackslash Mail}, normalmente em casos assim o arquivo \texttt{.gitignore} deve excluir do versionamento o diretório \texttt{vendor/} por questões de tamanho do projeto, porém nesse caso não o fizemos por conta de falta de familiaridade e compatibilidade do uso da ferramenta \texttt{composer}.
\begin{itemize}
    \item Envio automático de um código de 6 dígitos para o email informado pelo usuário.
    
    		Em \texttt{utilsMail.php} A função \texttt{generate2FACode()} gera um código de 6 dígitos e o salva em uma variável da sessão, ou seja, o usuário já deve ter sido aprovado no processo de login ou cadastro e uma sessão para ele já deve ter sido iniciada.
    		
    		A função \texttt{send2FACode()}, recebe o parâmetro obrigatório \textit{\$email} e é responsável por enviar o e-mail com o código de 6 dígitos gerado pela função anterior, ela recupera o código atravéz da sessão aberta para o usuário, recupera a chave da API SendGrid do arquivo \textit{.env} usando o \textit{Dotenv} e cria um corpo básico de e-mail com assunto e corpo e o envia ao e-mail passado no parâmetro.
    		
    		A função \texttt{verify2FACode()} recebe o parâmentro obrigatório \textit{\$inputCode} que é o código digitado pelo usuário, em seu fluxo, ela recupera da variável da sessão o código original e os compara retornando \textit{true} ou \textit{false} ela faz outras coisas como controlar o número máximo de tentativas permitidas e envia junto mensagens caso esse tempo seja ultrapassado. A fim de facilitar nosso trabalho e testes foi adicionado a essa função um código Bypass --- 000000 para evitar que sempre necessite aguardar o e-mail. Importante destacar que tal recurso é utilizado exclusivamente para fins de teste, não sendo aplicado em ambiente de produção.
    \item Verificação obrigatória antes de acessar o perfil.
    \item Bloqueio de tentativas inválidas por tentativas e por tempo de expiração do código enviado.
\end{itemize}

\section{Resultados Alcançados}
O 2FA foi integrado com sucesso ao sistema, fortalecendo a segurança do processo de autenticação e garantindo maior confiabilidade para os usuários e muito aprendizado no desenvolvimento.

% =====================================================================
\chapter{Sprint 4 – Reset de Senha}

\section{Objetivos}
A Sprint 4 teve como objetivo permitir que os usuários redefinissem sua senha de forma segura, utilizando validações JavaScript e envio de códigos pelo SendGrid.

\section{Atividades Executadas}
A funcionalidade foi desenvolvida com PHP, HTML, CSS, JavaScript, MySQL e SendGrid. Foram criadas quatro páginas PHP:

\subsection{\texttt{login.php}}
Inclui o link “Esqueceu sua senha?”, que redireciona o usuário para \texttt{emailCollect.php}.

\subsection{\texttt{emailCollect.php}}
Nesta página o usuário informa seu e-mail.  
Há validação em JavaScript e, no back-end, sanitização e verificação no banco de dados. Caso seja encontrado, sessões são criadas e o usuário é redirecionado.

\subsection{\texttt{autenticacao.php}}
A API SendGrid envia um código de seis dígitos. O código inserido pelo usuário é validado. Se correto, segue para \texttt{reset.php}.

\subsection{\texttt{reset.php}}
O usuário insere sua nova senha duas vezes.  
Validações em JavaScript garantem que as senhas coincidam.  
No back-end, o sistema localiza o usuário e atualiza a nova senha já criptografada com \texttt{password\_hash()}.

\section{Resultados Alcançados}
A Sprint 4 permitiu ao grupo aprender validações em JavaScript, sanitização segura, uso de APIs externas, manipulação de sessões e atualização de senhas com boas práticas de segurança.

% =====================================================================
\chapter{Conclusão}
%\addcontentsline{toc}{chapter}{Conclusão}

O desenvolvimento das quatro sprints resultou em um sistema funcional, seguro e estruturado, que evoluiu progressivamente em complexidade e maturidade. Na Sprint 1, foi construída a base do sistema de autenticação, incluindo cadastro, login e organização do banco de dados. Na Sprint 2, implementou-se um CRUD administrativo robusto, com hierarquia de permissões. A Sprint 3 ampliou a segurança com autenticação de dois fatores, integrando envio de códigos por e-mail. Por fim, a Sprint 4 consolidou um processo completo e seguro de redefinição de senha.

O grupo adquiriu experiência prática em desenvolvimento back-end, integração com banco de dados, segurança da informação, uso de APIs externas, boas práticas de organização e programação, validações, controle de sessões e arquitetura do sistema. O conjunto das sprints resultou em um sistema coerente, estável e alinhado com os objetivos da disciplina.

% =====================================================================

\appendix

\chapter{Links dos Repositórios e Materiais do Projeto}

Este apêndice reúne os principais links utilizados no desenvolvimento do sistema, incluindo o repositório responsável pelo código-fonte das quatro sprints e o quadro do Trello, onde foram organizadas as atividades, etapas e responsabilidades do grupo ao longo do projeto.

\section{Repositório MedSuam no GitHub}

A seguir, encontra-se o repositório que armazena o código produzido no decorrer das fases do desenvolvimento:

\begin{itemize}
    \item \textbf{Repositório Geral do Projeto:} \\
    \url{https://github.com/UnisuamWorkSpace/medSuam.git}
\end{itemize}

\section{Quadro do Trello}

O quadro do Trello foi utilizado para organização interna, registro das atividades, acompanhamento das tarefas por sprint e controle do fluxo de desenvolvimento. Seu acesso está disponível abaixo:

\begin{itemize}
    \item \textbf{Trello – Organização das Sprints:} \\
    \url{https://trello.com/b/ZZraZxb7/medsuam}
\end{itemize}

% =====================================================================
\end{document}